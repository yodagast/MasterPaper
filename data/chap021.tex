% !Mode:: "TeX:UTF-8"
%%% Local Variables:
%%% mode: latex
%%% TeX-master: t
%%% End:

%实体属性和关系路径知识库补全

\chapter{关系和属性特征计算}
\label{cha:kbc-lit-relation}
本论文提供一种结合关系路径特征和实体属性特征的知识库补全方法。
通过提取知识库中实体的关系路径特征和实体属性特征,构建模型,进行知识库的实体关系预测。
除此之外,本论文研究了基于学习排序算法进行知识库补全的技术,在传统基于分类器模型的基础上,
提出了非线性的排序算法,对知识库候选实体对进行排序计算,获得更优的候选实体对排序集合。

\section{关系路径特征计算}
\label{sec:relational-compute}
特征计算是知识库补全的特征抽取阶段,特征的数量和特征有效性是决定知识库补全效果的关键点之一。
给定一个完整的知识库,我们将知识库中的实体和关系分别看做为图中的顶点和边,
这样就能构建一个基于图模型的知识库补全系统。对于知识库中每种特定待预测的关系,
我们抽取对应关系下的实体对,将实体对的头实体和尾实体在图中进行随机游走\cite{Lao2012},
获得连接头实体和尾实体的关系路径特征,从而获得了知识库补全的关系路径特征。
给定一个目标关系r和这个关系对应的实体对集合
$$I_s=\{(s_j,t_j)|<s_j,t_j> \in KB\}$$


\subsection{关系路径类型集合}
\label{sec:relational-set}
我们期望找出给定关系下,所有连接头尾实体对的关系路径特征,最终获得一个关系路径类型的集合。
在知识图谱中,由于连接头尾实体对之间的路径数量很大,通常需要限定关系路径的长度,一般限定图中的关系路径长度在2-6之间。

对于能连接从头实体到尾实体的关系路径,我们记录这个关系路径的类型,作为模型预测的特征。
我们采用随机游走算法\cite{Lovsz1993RandomWO}计算选择从头实体到尾实体的关系路径,将这些路径集合作为关系路径类型特征集合。
例如对于实体对“北京师范大学,位于(locatedIn),北京”三元组,我们可以基于上述的关系路径抽取算法获得
“北京师范大学,位于(locatedIn),海淀区,位于(locatedIn),北京”、“北京师范大学,有校长(hasPresident),董奇,居住在(livesIn),北京”等多条路径来
获得关系“位于”对应的关系路径特征,从而生成一个关系路径集合,包括两条不同的关系路径类型$\{locatedIn\to locatedIn, hasPresident \to livesIn\}$。对于反方向的关系路径我们采用${livesIn}^{-1}$表示。

\subsection{关系路径特征向量}
获取关系路径类型集合(记为S(r))后,我们将给定关系r下的实体对$(s_j,t_j)$计算所有关系路径类型集合下的特征值。
如果实体对$(s_j,t_j)$在集合S(r)中的某个关系路径$s_i$存在,则我们可以将该实体对的对应关系路径类型的特征值记为1,
否则该实体对在这个关系路径类型下的特征值记为0,这种算法简化了不同关系类型在知识库中的权重,但是一些实验表明\cite{Gardner2014}
进行0-1二值化可以简化关系路径类型特征和特征值计算过程,同时对于实验结果的影响并不显著。为了能在不影响模型效果的前提下,
加速我们的计算过程,我们对自己的关系路径类型特征值计算进行了0-1二值化。

\section{实体属性特征计算}
\label{sec:literal-compute}
除了\ref{sec:relational-compute}中计算得到的关系路径类型,考虑到在知识库中任然存在大量的实体属性特征并未被使用,
本研究也对每个关系下的实体属性特征进行计算,获得这些实体属性特征下的特征值。
实体属性特征获取较为简单,只需要枚举知识库中存在的不同实体属性类型,即可这些实体属性类型作为实体特征集合。
实体属性特征的处理和计算是获取更有效特征、提升知识库补全算法的关键。

属性特征抽取过程较为复杂,不仅需要考虑不同维度的属性信息不一致问题,也要研究如何处理缺失值问题。
本论文通过枚举不同维度的实体属性特征,计算这些实体对在这些实体属性特征下的特征值。
首先对于每个实体,有着不同类型的描述特征,如一个人的信息,不仅有出生年月这种时间类型的信息,
也有年龄、性别等不同种类的属性信息。我们采用了对实体信息进行标准化的方法,
将实体对的头实体和尾实体这些不同属性特征进行归一化处理范围限定在[0,1]之间。
进行标准化后计算获得新的属性特征记为$V_l(h_i)$和$V_l (t_i)$,分别表示头实体和尾实体在所有熟悉特征下的实体属性特征向量,其中$h_i$ 和$t_i$表示给定关系l的第i个头实体和尾实体,同时我们对于头实体和尾实体进行相减计算获得$V_l(h_i-t_i )$属性值。除此之外,对于很多实体在不同特征上的缺失值,我们将缺失值进行了补0处理,期望获得更优结果。
如对于“北京师范大学”这个实体,我们抽取了他的“建校时间”、“占地面积”等所有的实体属性特征,
并计算每个实体对在这些特征下标准化后的特征值,而且将缺失属性特征补0。

在对每个关系下的实体对集合进行实体属性特征抽取计算后,我们获得了这个关系下头实体的实体属性特征、尾实体的实体属性特征、
对头尾实体进行归一化后的归一化实体属性特征,以及头实体和尾实体差值计算得到的实体属性特征。

通过将\ref{sec:relational-compute}关系路径特征和\ref{sec:literal-compute}抽取的实体属性特征进行结合,我们获得了基于关系路径类型和实体属性类型的
特征矩阵,并构建机器学习模型对生成的特征矩阵进行学习预测,获得知识库补全中的实体对的关系。

\section{关系路径和实体属性特征实验}

\subsection{YAGO知识库实验}
\label{cha:exp-literal}
本部分的效果展示了在YAGO知识库下进行知识库补全算法的结果,我们的方法和对比方法被分为两组。
其中,SFE\_literal(图中简称SFE\_lit)和PRA\_literal(图中简称PRA\_lit)是结合关系路径特征和实体属性特征的方法,其MAP、MRR和AUC预测结果最好,
总体的预测结果最稳定,效果最好,而未加实体属性特征的PRA和SFE特征较好,
使用表示学习的TransE和TransR模型效果在MAP中预测较差,
但是通过AUC和hit@1进行模型的二分类效果结果和基于符号逻辑的效果相似。
结果显示,加入实体属性特征后,结合实体属性特征和关系路径特征的补全技术,相比只采用关系路径特征的补全技术有更高的准确性,在进行排序相关的预测效果评测时,有较好的效果。
在采用二分类模型进行评价时候,无论是符号逻辑的路径排序相关算法还是表示学习的低维嵌入算法,都能有效的进行模型预测。表\ref{tab:addlabel-kbcExp-data}中展示了YAGO和Freebase知识库的基本数据情况。

% Table generated by Excel2LaTeX from sheet 'Sheet2'
\begin{table}[htbp]
  \centering
  \caption{Add caption}
    \begin{tabular}{|l|r|r|}
    \hline
    \textcolor[rgb]{ .141,  .161,  .18}{} & \multicolumn{1}{l|}{FB15K} & \multicolumn{1}{l|}{YAGO} \\
    \hline
    实体    & 14951 & 58130 \\
    \hline
    关系类型  & 1345  & 32 \\
    \hline
    实体属性类型 & 336   & 25 \\
    \hline
    属性三元组 & 25776 & 141606 \\
    \hline
    关系三元组 & 592213 & 499350 \\
    \hline
    训练集   & 483142 & 399480 \\
    \hline
    测试集   & 59071 & 49132 \\
    \hline
    \end{tabular}%
  \label{tab:addlabel-kbcExp-data}%
\end{table}%




相比只采用关系路径的知识库补全算法,结合实体属性特征和关系路径特征的知识库补全算法,
在YAGO数据集合上,结果有较大的提升,相比原来的模型,SFE-literal和PRA-literal在计算MAP时,都获得了近1\%的效果提升,模型提升效果统计显著。
基于上述实验可以获得如下结论:(1)预测知识库中新三元组通过结合关系路径特征和实体属性特征能更加的精确有效,无论是以分类为优化目标的分类模型,还是以排序为优化目标的排序问题。
(2)由于在YAGO知识库集合中,有更多的属性事实进行关系预测,采用特征组合,能获得更多的不同类型组合特征,达到更好的优化结果,因此,结合属性事实和关系事实进行预测是非常重要的。
(3)对于某些特殊的关系,进行标准化处理是非常有效的,但是并非对于所有的属性事实进行标准化有效,选择合适的属性特征和实体关系特征进行组合是十分必要的。
(4)本研究的实验结果表明,对于多数YAGO2中的关系来说,我们的属性事实不仅可以用来预测关系事实,
而且还能调整原来的关系特征的路径权重,使得模型预测更加合理。
因此,结合属性事实和更丰富的关系特征能获得更好的知识库补全结果。
% Table generated by Excel2LaTeX from sheet 'yago'
% yago literal result
\begin{table}[htbp]
  \centering
  \caption{YAGO知识库实体属性和关系路径特征模型预测结果}
    \begin{tabular}{|l|r|r|r|r|}
    \hline
          & \multicolumn{1}{l|}{MAP} & \multicolumn{1}{l|}{MRR} & \multicolumn{1}{l|}{AUC} & \multicolumn{1}{l|}{hit@1} \\
    \hline
    SFE-lit & 78.610\% & 100.000\% & 92.994\% & 45.328\% \\
    \hline
    PRA-lit & 78.170\% & 97.830\% & 93.496\% & 45.104\% \\
    \hline
    PRA   & 77.610\% & 97.830\% & 93.258\% & 45.509\% \\
    \hline
    SFE   & 77.360\% & 97.830\% & 93.225\% & 45.502\% \\
    \hline
    transE & 56.400\% & 91.200\% & 92.740\% & 45.642\% \\
    \hline
    transR & 46.470\% & 65.410\% & 92.986\% & 46.128\% \\
    \hline
    \end{tabular}%
  \label{tab:addlabel}%
\end{table}%


我们进一步分析了每个关系下关系路径和实体属性特征的权重,表\ref{tab:lit-rel-kbc}显示了在YAGO知识库中的三种关系:Export、GraduateFrom、HasAcademicAdvisor,通过结合实体属性特征和关系路径特征,学习获得的重要权重。相比于路径排序算法这种只使用关系特征进行预测的方式,
结合属性事实特征不仅仅能增加模型预测的全面性,将模型预测结果精度提高,同时也能调整逻辑回归算法中不同关系路径特征和不同属性事实特征的权重。通过将关系路径特征和属性事实特征进行结合,使得模型的预测结果更加可靠。
如表所示,我们分析关系“graduateFrom”可以发现,除了常见的关系路径特征:
isCitizenOf $\to$ isCitizenOf$^{-1}$ $\to$ livesIn $\to$ isLocatedIn$^{-1}$、
isAffiliatedTo$^{-1}$ $\to$ isCitizenOf $\to$ livesIn $\to$ isLocatedIn$^{-1}$
等。
一些重要的实体属性特征如:wasCreatedOnDate、happenedOnDate等都在关系预测结果中有着重要的作用。通过分析这些关系路径的权重我们可以发现,实体属性特征不仅能提高关系预测中的精度,有较高的准确性,同时也能调整关系路径的权重,产生更多的有表达力的关系路径特征。

\begin{table}[htbp]
  \centering
  \caption{YAGO知识库关系路径和实体属性特征比较}
    \begin{tabular}{cp{12.3cm}|p{3.7cm}|}
    \hline
    \multicolumn{3}{c}{Export} \\
    \hline
    \multirow{5}[2]{*}{PRA} & imports$\to$ hasMusicalRole$^{-1}$$\to$ hasMusicalRole &  \\
          & livesIn$^{-1}$$\to$ diedIn$\to$ imports  &  \\
          & livesIn$^{-1}$$\to$ wasBornIn$\to$ dealsWith$^{-1}$ $\to$ imports &  \\
          & isCitizenOf$^{-1}$$\to$ influences$^{-1}$ $\to$ isCitizenOf$^{-1}$ $\to$ imports &  \\
          & isCitizenOf$^{-1}$$\to$ wasBornIn $\to$ dealsWith$^{-1}$ $\to$ imports &  \\
    \hline
    \multirow{5}[2]{*}{PRA-lit} & hasCapital $\to$ hasCapital$^{-1}$ $\to$ exports & hasGini \\
          & imports $\to$ hasMusicalRole$^{-1}$ $\to$ hasMusicalRole$^{-1}$ & hasInfaltion \\
          & isCitizenOf$^{-1}$ $\to$ wasBornIn $\to$ hasCapital$^{-1}$ $\to$ exports & hasEconomicGrowth \\
          & exports $\to$ hasMusicalRole$^{-1}$ $\to$ hasMusicalRole & hasPoverty \\
          & isInterestedIn$^{-1}$$\to$ wasBornIn $\to$ hasCapital$^{-1}$ $\to$ exports& wasDestroyedOnDate \\
    \hline
    \multicolumn{3}{c}{GraduateFrom} \\
    \hline
    \multirow{5}[2]{*}{PRA} & isCitizenOf $\to$ isCitizenOf$^{-1}$ $\to$ livesIn $\to$ isLocatedIn$^{-1}$&  \\
          & diedIn $\to$ happenedIn$^{-1}$ $\to$ participatedIn$^{-1}$ $\to$ isLocatedIn$^{-1}$ &  \\
          & hasWebsite $\to$ hasWebsite$^{-1}$ $\to$ livesIn $\to$ isLocatedin$^{-1}$ &  \\
          & isAffiliatedTo $\to$ isAffiliatedTo$^{-1}$ $\to$ isCitizenOf $\to$ isLocatedIn$^{-1}$ &  \\
          & isAffiliatedTo$^{-1}$ $\to$ isCitizenOf $\to$ livesIn $\to$ isLocatedIn$^{-1}$ &  \\
    \hline
    \multirow{5}[2]{*}{PRA-lit} & isAffiliatedTo $\to$ isAffiliatedTo$^{-1}$ $\to$ isCitizenOf $\to$ isLocatedIn$^{-1}$& happenedOnDate\\
          & hasAcademicAdvisor $\to$ hasAcademicAdvisor$^{-1}$ $\to$ graduatedFrom &wasDestroyedOnDate \\
          & hasWebsite $\to$ hasWebsite$^{-1}$ $\to$ livesIn $\to$ isLocatedin$^{-1}$ & wasDestroyedOnDate\\
          & isAffiliatedTo $\to$ isAffiliatedTo$^{-1}$ $\to$ isLeaderOf $\to$ isLocatedIn$^{-1}$ & wasCreatedOnDate \\
          & isCitizenOf $\to$ isCitizenOf$^{-1}$ $\to$ livesIn $\to$ isLocatedIn$^{-1}$ & wasBornOnDate \\
    \hline
    \multicolumn{3}{c}{HasAcademicAdvisor} \\
    \hline
    \multirow{5}[2]{*}{PRA} & wasBornIn $\to$ happenedIn$^{-1}$ $\to$ participatedIn$^{-1}$ $\to$ livesIn$^{-1}$ &  \\
          & diedIn $\to$ hasCapital$^{-1}$ $\to$ isLocatedIn$^{-1}$ $\to$ diedIn$^{-1}$ &  \\
          & worksAt $\to$ graduatedFrom$^{-1}$ $\to$ livesIn $\to$ wasBornIn$^{-1}$ &  \\
          & hasAcademicAdvisor $\to$ hasAcademicAdvisor$^{-1}$$\to$ influences$^{-1}$ $\to$ hasAcademicAdvisor &  \\
          & livesIn$\to$ diedIn $\to$ hasAcademicAdvisor &  \\
    \hline
    \multirow{5}[2]{*}{PRA-lit} & hasGender $\to$ hasGender$^{-1}$ & wasDestroyedOnDate \\
          & worksAt $\to$ graduatedFrom$^{-1}$ $\to$ livesIn $\to$ wasBornIn$^{-1}$ & hasHeight \\
          & hasAcademicAdvisor $\to$ hasAcademicAdvisor$^{-1}$ $\to$ diedIn $\to$ diedIn$^{-1}$ & hasHeight \\
          & diedIn $\to$ diedIn$^{-1}$ $\to$ livesIn $\to$ livesIn$^{-1}$ & wasBornOnDate \\
          & graduatedFrom $\to$ worksAt $\to$ worksAt$^{-1}$ $\to$ worksAt  & diedOnDate \\
    \hline
    \hline
    \end{tabular}%
  \label{tab:lit-rel-kbc}%
\end{table}%
% Table generated by Excel2LaTeX from sheet '工作表1'


\subsection{Freebase知识库实验}

% fb15k实验结果
\begin{table}[htbp]
  \centering
  \caption{Freebase知识库实体属性和关系路径特征模型预测结果}
    \begin{tabular}{|l|r|r|r|r|}
    \hline
          & \multicolumn{1}{l|}{MAP} & \multicolumn{1}{l|}{MRR} & \multicolumn{1}{l|}{AUC} & \multicolumn{1}{l|}{hit@1} \\
    \hline
    transR & 66.440\% & 95.500\% & 86.182\% & 44.145\% \\
    \hline
    transE & 72.910\% & 98.650\% & 89.118\% & 47.348\% \\
    \hline
    SFE   & 86.490\% & 98.650\% & 97.087\% & 55.559\% \\
    \hline
    SFE-lit & 86.560\% & 100.000\% & 97.140\% & 55.636\% \\
    \hline
    PRA   & 86.930\% & 98.200\% & 97.021\% & 55.527\% \\
    \hline
    PRA-lit & 87.140\% & 100.000\% & 97.108\% & 55.606\% \\
    \hline
    \end{tabular}%
  \label{tab:addlabel}%
\end{table}%
本部分的效果展示了在Freebase知识库下进行知识库补全算法的结果,我们的方法和对比方法被分为两组,表\ref{tab:addlabel-fb}展示了在不同评价指标下Freebase知识库中部分关系的预测结果。
其中,SFE-literal和PRA-literal是结合关系路径特征和实体属性特征的方法,其MAP、MRR和AUC预测结果最好,
尤其在采用AUC进行模型评价时候,相比于表示学习方法,基于符号逻辑的知识库补全算法特征有非常高的提升。从基于MAP和MRR的评价指标来看,加入实体属性特征的总体预测结果最稳定,效果最好,而未加实体属性特征的PRA和SFE特征较好,
使用表示学习的TransE和TransR模型效果在MAP中预测较差,
结果显示,加入实体属性特征后,结合实体属性特征和关系路径特征的补全技术,相比只采用关系路径特征的补全技术有更高的准确性,在进行排序相关的预测效果评测时,有较好的效果。
在采用二分类模型进行评价时候,无论是符号逻辑的路径排序相关算法还是表示学习的低维嵌入算法,都能有效的进行模型预测。

总体上看,使用MAP、MRR、AUC以及hit@1评价指标时,基于符号逻辑的知识库补全算法效果较基于表示学习的算法性能有较大的提升,而结合实体属性特征和结合关系路径特征的知识库补全算法相比仅仅使用关系路径特征的模型效果更好。这说明无论采用分类作为模型的优化方向,还是采用实体对排序的序列作为知识库补全优化目标,使用符号逻辑相比于近似的表示学习方法效果都有明显的提升。


分关系来看,从表\ref{tab:addlabel-fb-map}可以分析发现,在Freebase的15种关键关系中,
我们通过采用结合关系路径特征和实体属性特征的知识库补全算法,相比于只采用关系路径的知识库补全算法,我们的模型效果在MAP排序指标上有很大的提升。如/tv/tv/genreprograms、/filmactor/filmfilmperformancefilm 和 /film/director/film 、/media/commonnetflix/genretitles等关系使用MAP评测都有较大的模型效果提升。

% Table generated by Excel2LaTeX from sheet 'FB15K'
\begin{table}[htbp]
  \centering
  \caption{FB15K部分关系MAP得分}
    \begin{tabular}{|l|r|r|r|r|r|r|}
    \hline
    MAP-score & \multicolumn{1}{l|}{PRA-lit  } & \multicolumn{1}{l|}{PRA} & \multicolumn{1}{l|}{SFE-lit} & \multicolumn{1}{l|}{sfe} & \multicolumn{1}{l|}{transE} & \multicolumn{1}{l|}{transR} \\
    \hline
    /people/person/profession & 12.150\% & 11.902\% & 12.359\% & 12.096\% & 77.134\% & 71.784\% \\
    \hline
    /film/actor/film/film/performance/film & 28.309\% & 26.179\% & 22.121\% & 20.560\% & 87.007\% & 78.431\% \\
    \hline
    /media/common/netflix/genre/titles & 50.135\% & 48.281\% & 49.919\% & 49.546\% & 63.788\% & 65.603\% \\
    \hline
    /people/ethnicity/people & 57.143\% & 57.143\% & 57.143\% & 57.143\% & 60.402\% & 50.396\% \\
    \hline
    /film/film/genre/films/in/this/genre & 57.285\% & 55.650\% & 41.435\% & 41.828\% & 68.762\% & 64.254\% \\
    \hline
    /music/instrument/instrumentalists & 58.268\% & 58.268\% & 58.268\% & 58.268\% & 53.702\% & 52.452\% \\
    \hline
    /music/genre/artists & 69.514\% & 69.514\% & 69.514\% & 69.514\% & 77.745\% & 65.328\% \\
    \hline
    /location/location/time/zones & 77.479\% & 77.130\% & 77.479\% & 77.479\% & 82.790\% & 72.245\% \\
    \hline
    /tv/tv/genre/programs & 80.952\% & 80.952\% & 80.952\% & 80.952\% & 78.140\% & 60.484\% \\
    \hline
    /people/person/nationality & 81.119\% & 79.616\% & 81.774\% & 80.770\% & 75.448\% & 73.757\% \\
    \hline
    /people/cause/of/death/people & 82.653\% & 82.653\% & 82.653\% & 82.653\% & 52.979\% & 44.508\% \\
    \hline
    /music/record/label/artist & 83.704\% & 83.704\% & 83.704\% & 83.704\% & 55.037\% & 41.867\% \\
    \hline
    /film/production/company/films & 85.561\% & 85.561\% & 85.561\% & 85.561\% & 68.758\% & 59.189\% \\
    \hline
    /film/director/film & 100.000\% & 100.000\% & 100.000\% & 100.000\% & 94.077\% & 88.385\% \\
    \hline
    /film/film/directed/by & 100.000\% & 100.000\% & 100.000\% & 100.000\% & 95.254\% & 85.679\% \\
    \hline
    \end{tabular}%
  \label{tab:addlabel-fb-map}%
\end{table}%


我们以关系peoplepersonnationality和关系filmactorfilmfilmperformancefilm进行分析,研究关系路径特征和实体属性特征在路径排序算法中的特征表现。通过分析MAP指标来看,TransE和TransR这两个模型在各个关系上的效果较为稳定,符号逻辑相关算法在这些不同的关系上效果差别较大,通过分析相关的关系路径和实体属性权重,我们可以发现,这些关系路径特征相对于YAGO知识库来说,关系路径权重区别并不明显,甚至在结合实体属性特征后,对于filmactorfilmfilmperformancefilm这个关系来说,所有的关系路径和实体属性路径特征综合从44120减少到42787种特征。
说明基于随机游走抽取的关系路径特征并不能很好的区分Freebase中实体和实体之间的关系,
但是当加入关系路径特征后,能很好的惩罚这些加入区分度不好的关系路径特征。

其次尽管部分关系相对于表示学习来说,预测结果的效果较差,但是,基于图模型的知识库补全系统中,大部分关系的预测相对于基于表示学习效果提升很多。这一方面说明基于表示学习是一种近似的模型预测,而基于符号逻辑的关系路径学习排序则是一种精确而有效的学习方式。另一方面也说明,
在知识库构建的图模型中,部分关系较为稀疏的实体之间使用表示学习,预测结果较为理想,
而采用结合实体属性和关系路径的符号逻辑补全算法,则能基于有效的关系路径,获得十分有用的预测模型结果。

其次,由于Freebase知识库总共包含超过1345种不同类型的关系,这些关系覆盖了电影、人种、电视剧、音乐等不同类型的关系,这些不同的关系中部分关系具有对称性,一些关系和关系之间差别很大,
如何能将这些不同类型互不影响或者相互对称的关系区分出来,构建一个合理的图,基于这种合理的图模型进行关系路径的抽取和实体属性特征的计算,是未来研究的重要研究步骤。





