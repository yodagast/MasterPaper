% !Mode:: "TeX:UTF-8"
%%% Local Variables:
%%% mode: latex
%%% TeX-master: t
%%% End:

\chapter{关系和属性特征计算}
\label{cha:kbc-lit-relation}
本论文提供一种结合关系路径特征和实体属性特征的知识库补全方法。
通过提取知识库中实体的关系路径特征和实体属性特征,构建模型,进行知识库的实体关系预测。
除此之外,本论文研究了基于学习排序算法进行知识库补全的技术,在传统基于分类器模型的基础上,
提出了非线性的排序算法,对知识库候选实体对进行排序计算,获得更优的候选实体对排序集合。

\section{关系路径特征计算}
\label{sec:relational-compute}
特征计算是知识库补全的特征抽取阶段,特征的数量和特征有效性是决定知识库补全效果的关键点之一。
给定一个完整的知识库,我们将知识库中的实体和关系分别看做为图中的顶点和边,
这样就能构建一个基于图模型的知识库补全系统。对于知识库中每种特定待预测的关系,
我们抽取对应关系下的实体对,将实体对的头实体和尾实体在图中进行随机游走\cite{Lao2012},
获得连接头实体和尾实体的关系路径特征,从而获得了知识库补全的关系路径特征。
给定一个目标关系r和这个关系对应的实体对集合
$$I_s=\{(s_j,t_j)|<s_j,t_j> \in KB\}$$


\subsection{关系路径类型集合}
\label{sec:relational-set}
我们期望找出给定关系下,所有连接头尾实体对的关系路径特征,最终获得一个关系路径类型的集合。
在知识图谱中,由于连接头尾实体对之间的路径数量很大,通常需要限定关系路径的长度,一般限定图中的关系路径长度在2-6之间。

对于能连接从头实体到尾实体的关系路径,我们记录这个关系路径的类型,作为模型预测的特征。
我们采用随机游走算法\cite{Lovsz1993RandomWO}计算选择从头实体到尾实体的关系路径,将这些路径集合作为关系路径类型特征集合。
例如对于实体对“北京师范大学,位于(locatedIn),北京”三元组,我们可以基于上述的关系路径抽取算法获得
“北京师范大学,位于(locatedIn),海淀区,位于(locatedIn),北京”、“北京师范大学,有校长(hasPresident),董奇,居住在(livesIn),北京”等多条路径来
获得关系“位于”对应的关系路径特征,从而生成一个关系路径集合,包括两条不同的关系路径类型$\{locatedIn\to locatedIn, hasPresident \to livesIn\}$。对于反方向的关系路径我们采用${livesIn}^{-1}$表示。

\subsection{关系路径特征向量}
获取关系路径类型集合(记为S(r))后,我们将给定关系r下的实体对$(s_j,t_j)$计算所有关系路径类型集合下的特征值。
如果实体对$(s_j,t_j)$在集合S(r)中的某个关系路径$s_i$存在,则我们可以将该实体对的对应关系路径类型的特征值记为1,
否则该实体对在这个关系路径类型下的特征值记为0,这种算法简化了不同关系类型在知识库中的权重,但是一些实验表明\cite{Gardner2014}
进行0-1二值化可以简化关系路径类型特征和特征值计算过程,同时对于实验结果的影响并不显著。为了能在不影响模型效果的前提下,
加速我们的计算过程,我们对自己的关系路径类型特征值计算进行了0-1二值化。

\section{实体属性特征计算}
\label{sec:literal-compute}
除了\ref{sec:relational-compute}中计算得到的关系路径类型,考虑到在知识库中任然存在大量的实体属性特征并未被使用,
本研究也对每个关系下的实体属性特征进行计算,获得这些实体属性特征下的特征值。
实体属性特征获取较为简单,只需要枚举知识库中存在的不同实体属性类型,即可这些实体属性类型作为实体特征集合。
实体属性特征的处理和计算是获取更有效特征、提升知识库补全算法的关键。

属性特征抽取过程较为复杂,不仅需要考虑不同维度的属性信息不一致问题,也要研究如何处理缺失值问题。
本论文通过枚举不同维度的实体属性特征,计算这些实体对在这些实体属性特征下的特征值。
首先对于每个实体,有着不同类型的描述特征,如一个人的信息,不仅有出生年月这种时间类型的信息,
也有年龄、性别等不同种类的属性信息。我们采用了对实体信息进行标准化的方法,
将实体对的头实体和尾实体这些不同属性特征进行归一化处理范围限定在[0,1]之间。
进行标准化后计算获得新的属性特征记为$V_l(h_i)$和$V_l (t_i)$,分别表示头实体和尾实体在所有熟悉特征下的实体属性特征向量,其中$h_i$ 和$t_i$表示给定关系l的第i个头实体和尾实体,同时我们对于头实体和尾实体进行相减计算获得$V_l(h_i-t_i )$属性值。除此之外,对于很多实体在不同特征上的缺失值,我们将缺失值进行了补0处理,期望获得更优结果。
如对于“北京师范大学”这个实体,我们抽取了他的“建校时间”、“占地面积”等所有的实体属性特征,
并计算每个实体对在这些特征下标准化后的特征值,而且将缺失属性特征补0。

在对每个关系下的实体对集合进行实体属性特征抽取计算后,我们获得了这个关系下头实体的实体属性特征、尾实体的实体属性特征、
对头尾实体进行归一化后的归一化实体属性特征,以及头实体和尾实体差值计算得到的实体属性特征。

通过将\ref{sec:relational-compute}关系路径特征和\ref{sec:literal-compute}抽取的实体属性特征进行结合,我们获得了基于关系路径类型和实体属性类型的
特征矩阵,并构建机器学习模型对生成的特征矩阵进行学习预测,获得知识库补全中的实体对的关系。 