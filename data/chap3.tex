% !Mode:: "TeX:UTF-8"
%%% Local Variables:
%%% mode: latex
%%% TeX-master: t
%%% End:

\chapter{基于逻辑符号的知识库补全实验}
\label{cha:kbc-exp}
考虑到仅仅有部分的知识库系统同时拥有关系型三元组和属性型三元组,本文构建了一个基于YAGO的知识库补全实验。
\ref{cha:exp-literal}研究了如何在仅有关系路径特征的基础上,增加更多的实体属性特征,构建更好的知识库补全系统。
\ref{cha:exp-relational}研究了如何在传统的基于分类模型的基础上,采用基于学习排序的算法,构建实体对排序的秩序最优结果。


\section{YAGO知识库}
本研究构建了一个面向YAGO的知识库补全实例。YAGO是一个从维基百科上抽取的、包含地理名词、WordNet\cite{Kasneci2008TheYA}等数据的知识库。
YAGO将WordNet的词汇定义与维基百科的分类体系进行了融合集成\cite{Suchanek2008YAGOAL},使得YAGO具有更加丰富的实体分类体系。YAGO还考虑了时间和空间知识,为很多知识条目增加了时间和空间维度的属性描述。目前,YAGO包含1.2亿条三元组知识。YAGO是IBM Watson\cite{Ferrucci2010BuildingWA}的后端知识库之一。
而YAGO2是YAGO的一个实例,当前YAGO2包括超过千万的实体和超过1.2亿的实体知识,我们使用了其中实体的关系型三元组和属性型三元组共有4,484,914条、
37种关系型三元组的事实描述,同时有3,353,659条、35种属性性三元组的事实描述。

本研究中对于每种关系下的三元组,基于局部封闭世界假设,生成正负三元组。
对于每个正实体对,生成10个负实体对,其中5个随机替换头实体对,5个随机替换尾实体对。
在\ref{cha:exp-literal}中我们使用$YAGO_{all}$数据集。此外,我们还考虑到很多实体对缺少实体属性特征数据,在结合关系路径特征和实体属性特征中预测效果提升不明显,
因而采用算法过滤掉$YAGO_{all}$中缺失实体属性特征的实体对,构建了第二个知识图谱补全数据集合,称之为$YAGO_{lit}$。

\ref{cha:exp-relational}评测了路径排序算法(PRA)、子图特征抽取(SFE)两种常见的模型和基于学习排序的路径排序算法(rankPRA)、基于学习排序的子图特征抽取算法(rankSFE)在YAGO2的37种关系中的预测结果。
其中子图特征抽取是路径排序算法的一种拓展,相比传统路径排序算法,子图特征抽取算法能计算在特征计算上获得更多的路径特征。
基于学习排序算法的路径排序和子图特征抽取则在当前研究的基础上,改进模型的预测规则,将基于分类或回归的预测算法转化为基于排序的模型,
实验表明这样能更有效进行模型预测。模型评测的指标选取信息检索领域常见的指标MAP和MRR。

\section{关系路径特征和属性事实特征结合}
\label{cha:exp-literal}
本部分的效果展示了$YAGO_{all}$和$YAGO_{lit}$两种不同方法的结果,方法和对比方法被分为两组,PRA、$IRL_{pra}$ 、$IRL_{pra}^{nor}$和SFE、$IRL_{SFE}$ 、$IRL_{SFE}^{nor}$,在同一个组中使用相同的关系路径特征,我们在表格3中使用比较了不同的方法的平均精度即MAP在不同方法下的计算结果。
我们的结果显示,结合实体属性特征和关系路径特征的补全技术,相比只采用关系路径特征的补全技术有更高的准确性。


% Table generated by Excel2LaTeX from sheet 'Sheet1'
\begin{table}[htbp]
  \centering
  \caption{关系路径和实体属性特征的知识库补全的MAP结果}
    \begin{tabular}{|l|c|c|c|c|c|c|}
    \hline
    \multicolumn{1}{|c|}{} & \multicolumn{1}{l|}{PRA} & \multicolumn{1}{l|}{$IRL_{pra}$} & \multicolumn{1}{l|}{$IRL_{PRA}^{nor}$} & \multicolumn{1}{l|}{SFE} & \multicolumn{1}{l|}{$IRL_{sfe}$} & \multicolumn{1}{l|}{$IRL_{sfe}^{nor}$} \\
    \hline
    $YAGO_{all}$ & 0.4413 & 0.4635 & 0.4756 & 0.4643 & 0.4756 & 0.4729 \\
    \hline
    $YAGO_{lit}$ & 0.6128 & 0.6643 & 0.6632 & 0.6211 & 0.6823 & 0.669 \\
    \hline
    \end{tabular}%
  \label{tab:addlabel}%
\end{table}%


如表\ref{tab:addlabel}结果显示,结合实体属性特征的知识图谱补全方法相比于只基于路径特征的知识图谱补全方法,结果有较大的提升。在$YAGO_{all}$数据集合上,
$IRL_{pra}^{nor}$相比其他模型,有着较大的提升结果,在$YAGO_{lit}$数据集合上,结果显示$IRL_{SFE}$和$IRL_{SFE}^{nor}$ 都获得了非常显著的结果提升。同时$IRL_{pra}$获得了5\%的结果提升,而$IRL_SFE$ 获得了6\%的提升。
基于上述实验可以获得如下结论:预测知识库中新三元组通过结合关系路径特征和实体属性特征能更加的精确有效。
其次,由于$YAGO_{lit}$相比$YAGO_{all}$数据集合,有更加的多的属性事实进行关系预测,因此,结合属性事实和关系事实进行预测是非常重要的。
第三,对于某些特殊的关系,进行标准化处理是非常有效的,但是并非对于所有的属性事实进行标准化有效。
本发明的实验结果表明,对于多数YAGO2中的关系来说,我们的属性事实不仅可以用来预测关系事实,
而且还能调整原来的关系特征的路径权重,使得模型预测更加合理。
因此,结合属性事实和更丰富的关系特征能获得更好的知识库补全结果。

表\ref{tab:lit-rel-kbc}结果显示了三种关系的属性事实特征和关系路径特征。相比于路径排序算法这种只使用关系特征进行预测的方式,
结合属性事实特征不仅仅能增加模型预测的全面性,将模型预测结果精度提高,同时也能调整逻辑回归算法中不同关系路径特征和
不同属性事实特征的权重。通过将关系路径特征和属性事实特征进行结合,使得模型的预测结果更加可靠。
如表所示,我们分析关系“graduateFrom”可以发现,除了常见的关系路径特征:
isCitizenOf $\to$ isCitizenOf$^{-1}$ $\to$ livesIn $\to$ isLocatedIn$^{-1}$等。
一些重要的实体属性特征如:wasCreatedOnDate、happenedOnDate等都在关系预测结果中有着重要的作用。

\begin{table}[htbp]
  \centering
  \caption{属性特征和关系路径特征比较}
    \begin{tabular}{cp{12.6cm}|p{3.4cm}|}
    \toprule
    \multicolumn{3}{c}{Export} \\
    \midrule
    \multirow{5}[2]{*}{PRA} & imports$\to$ hasMusicalRole$^{-1}$$\to$ hasMusicalRole &  \\
          & livesIn$^{-1}$$\to$ diedIn$\to$ imports  &  \\
          & livesIn$^{-1}$$\to$ wasBornIn$\to$ dealsWith$^{-1}$ $\to$ imports &  \\
          & isCitizenOf$^{-1}$$\to$ influences$^{-1}$ $\to$ isCitizenOf$^{-1}$ $\to$ imports &  \\
          & isCitizenOf$^{-1}$$\to$ wasBornIn $\to$ dealsWith$^{-1}$ $\to$ imports &  \\
    \midrule
    \multirow{5}[2]{*}{IRL} & hasCapital $\to$ hasCapital$^{-1}$ $\to$ exports & hasGini \\
          & imports $\to$ hasMusicalRole$^{-1}$ $\to$ hasMusicalRole$^{-1}$ & hasInfaltion \\
          & isCitizenOf$^{-1}$ $\to$ wasBornIn $\to$ hasCapital$^{-1}$ $\to$ exports & hasEconomicGrowth \\
          & exports $\to$ hasMusicalRole$^{-1}$ $\to$ hasMusicalRole & hasPoverty \\
          & isInterestedIn$^{-1}$$\to$ wasBornIn $\to$ hasCapital$^{-1}$ $\to$ exports& wasDestroyedOnDate \\
    \midrule
    \multicolumn{3}{c}{GraduateFrom} \\
    \midrule
    \multirow{5}[2]{*}{PRA} & isCitizenOf $\to$ isCitizenOf$^{-1}$ $\to$ livesIn $\to$ isLocatedIn$^{-1}$&  \\
          & diedIn $\to$ happenedIn$^{-1}$ $\to$ participatedIn$^{-1}$ $\to$ isLocatedIn$^{-1}$ &  \\
          & hasWebsite $\to$ hasWebsite$^{-1}$ $\to$ livesIn $\to$ isLocatedin$^{-1}$ &  \\
          & isAffiliatedTo $\to$ isAffiliatedTo$^{-1}$ $\to$ isCitizenOf $\to$ isLocatedIn$^{-1}$ &  \\
          & isAffiliatedTo$^{-1}$ $\to$ isCitizenOf $\to$ livesIn $\to$ isLocatedIn$^{-1}$ &  \\
    \midrule
    \multirow{5}[2]{*}{IRL} & isAffiliatedTo $\to$ isAffiliatedTo$^{-1}$ $\to$ isCitizenOf $\to$ isLocatedIn$^{-1}$& happenedOnDate\\
          & hasAcademicAdvisor $\to$ hasAcademicAdvisor$^{-1}$ $\to$ graduatedFrom &wasDestroyedOnDate \\
          & hasWebsite $\to$ hasWebsite$^{-1}$ $\to$ livesIn $\to$ isLocatedin$^{-1}$ & wasDestroyedOnDate\\
          & isAffiliatedTo $\to$ isAffiliatedTo$^{-1}$ $\to$ isLeaderOf $\to$ isLocatedIn$^{-1}$ & wasCreatedOnDate \\
          & isCitizenOf $\to$ isCitizenOf$^{-1}$ $\to$ livesIn $\to$ isLocatedIn$^{-1}$ & wasBornOnDate \\
              \multicolumn{3}{c}{hasCurrency} \\
    \midrule
    \multicolumn{3}{c}{HasAcademicAdvisor} \\
    \midrule
    \multirow{5}[2]{*}{PRA} & wasBornIn $\to$ happenedIn$^{-1}$ $\to$ participatedIn$^{-1}$ $\to$ livesIn$^{-1}$ &  \\
          & diedIn $\to$ hasCapital$^{-1}$ $\to$ isLocatedIn$^{-1}$ $\to$ diedIn$^{-1}$ &  \\
          & worksAt $\to$ graduatedFrom$^{-1}$ $\to$ livesIn $\to$ wasBornIn$^{-1}$ &  \\
          & hasAcademicAdvisor $\to$ hasAcademicAdvisor$^{-1}$$\to$ influences$^{-1}$ $\to$ hasAcademicAdvisor &  \\
          & livesIn$\to$ diedIn $\to$ hasAcademicAdvisor &  \\
    \midrule
    \multirow{5}[2]{*}{IRL} & hasGender $\to$ hasGender$^{-1}$ & wasDestroyedOnDate \\
          & worksAt $\to$ graduatedFrom$^{-1}$ $\to$ livesIn $\to$ wasBornIn$^{-1}$ & hasHeight \\
          & hasAcademicAdvisor $\to$ hasAcademicAdvisor$^{-1}$ $\to$ diedIn $\to$ diedIn$^{-1}$ & hasHeight \\
          & diedIn $\to$ diedIn$^{-1}$ $\to$ livesIn $\to$ livesIn$^{-1}$ & wasBornOnDate \\
          & graduatedFrom $\to$ worksAt $\to$ worksAt$^{-1}$ $\to$ worksAt  & diedOnDate \\
    \midrule
    \bottomrule
    \end{tabular}%
  \label{tab:lit-rel-kbc}%
\end{table}%
% Table generated by Excel2LaTeX from sheet '工作表1'

\section{基于学习排序的知识库补全}
\label{cha:exp-relational}

展示了基于YAGO2的总体MAP和MRR进行模型评价的结果。我们可以看到基于学习排序的知识库补全技术相比传统的路径排序算法在MAP上有很大的提升,
基于学习排序的算法相比传统的算法在YAGO2数据集上有近50\%的效果提升;而四种方法在MRR指标上效果相当,
基于学习排序算法的MRR并未比传统路径排序算法有显著下降。

我们详细分析了37种YAGO2中关系的MAP指标,并在表 \ref{rank}展示了部分关系的MAP值。
分析可以发现,大部分关系采用学习排序算法后,预测结果有较大的提高,而在不同的关系类型预测中,
MAP差别较大,如:“playFor”和“isConnectedTo”等关系预测有较大的提高,而在“isInterestedIn”等关系中关系预测提升较差。
总体来说,有超过30种关系的MAP预测获得了显著的提升,只有不到5种关系MAP预测结果并未提升或略有下降,
这样的实验说明基于学习排序算法的知识库补全技术相比传统的分类回归打分模型有非常大的效果提升。


\begin{table}[htbp]
  \centering
  \caption{学习排序补全算法部分关系MAP值}
    \begin{tabular}{|l|r|r|r|r|}
    \hline
    relation & \multicolumn{1}{l|}{PRA} & \multicolumn{1}{l|}{SFE} & \multicolumn{1}{l|}{rankPRA} & \multicolumn{1}{l|}{rankSFE} \\
    \hline
    actedIn & 0.3379  & 0.3496  & 0.6222  & 0.6252  \\
    \hline
    created & 0.2523  & 0.2532  & 0.3089  & 0.3128  \\
    \hline
    dealsWith & 0.1729  & 0.1411  & 0.1265  & 0.1572  \\
    \hline
    graduatedFrom & 0.2646  & 0.2726  & 0.5607  & 0.5726  \\
    \hline
    hasCapital & 0.5637  & 0.6014  & 0.7275  & 0.7304  \\
    \hline
    hasChild & 0.5004  & 0.5078  & 0.6674  & 0.6757  \\
    \hline
    influences & 0.2946  & 0.2932  & 0.5771  & 0.5836  \\
    \hline
    isAffiliatedTo & 0.6364  & 0.6538  & 0.7816  & 0.7840  \\
    \hline
    playsFor & 0.6538  & 0.6606  & 0.8020  & 0.8036  \\
    \hline
    wasBornIn & 0.3661  & 0.3742  & 0.6053  & 0.6070  \\
    \hline
    worksAt & 0.2343  & 0.2281  & 0.5022  & 0.5014  \\
    \hline
    wroteMusicFor & 0.3488  & 0.3621  & 0.6144  & 0.6161  \\
    \hline
    \end{tabular}%
  \label{rank}%
\end{table}%





