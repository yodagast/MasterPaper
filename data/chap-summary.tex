% !Mode:: "TeX:UTF-8"
%%% Local Variables:
%%% mode: latex
%%% TeX-master: t
%%% End:

% 总结和展望
\chapter{总结和展望}
\label{cha:kbc-summary}
随着知识库在信息检索、手机助手、对话系统、人工智能等各个领域的广泛发展,知识库的完整性和真实性越来越重要,而如何对知识库进行补全,验证实体和实体之间关系的真实性,都是提高知识库可用性的重要问题。本研究从基于路径排序算法的知识库补全进行研究,选择实体和实体之间的关系路径、头尾实体中不同的属性值进行组合、并构建模型、选择合理有效的目标函数进行知识库补全。同时,本研究也结合了一些新的深度神经网络知识,将不同的特征进行深度融合,获得了更好的知识库补全效果。然而,基于符号逻辑的路径排序算法,补全效果和知识库图的拓扑结构有关,如何能在稀疏的图中进行知识库补全,如何更好的结合关系路径特征和实体属性特征以及基于表示学习获得的实体、关系向量表示特征,是未来进行知识库补全工作的重点。

\section{论文工作总结}
知识库补全算法是知识库构建和知识库应用的重要桥梁,基于信息抽取获得的三元组,需要通过知识库补全技术提高知识库的可用性,进而才能提供各种有效的知识库应用系统。本论文从知识库补全的特征构建开始,结合实体属性特征和关系路径特征,将不同类型的关系路径特征,不同类别的实体属性特征结合,能提高知识库补全的模型的可预测性,可拓展性。此外,本研究拓展了原有的知识库补全模型,将原来基于pointwise的二分类模型,拓展为基于学习排序算法的pairwise模型,提供了更多的模型优化方法,作为原有基于逻辑回归中的log损失函数,或者基于支持向量机的hinge损失函数的补充。这些不同的损失函数作为机器学习的目标,在进行知识库补全工作中,可以依据具体任务不同而选择合适的模型进行预测。

本研究的实验发现了很多重要的结论,能为后续的知识库补全工作提供研究思路。(1)知识库中的关系路径系统是稀疏的,将稀疏的关系路径处理好来预测实体和实体之间的关系,是知识库补全的重要步骤。(2)基于符号逻辑和关系路径进行知识库补全,预测效果和图的拓扑结构密切相关。对于一些关系的实体,这些实体和其他实体之间关系很少,是一类比较孤立的顶点。这导致了基于随机游走过程中,部分实体对很难抽取有表达力的关系路径特征,同样对于部分实体对,他们的实体属性特征也很少,这样会导致知识库补全算法很难进行模型预测。如何结合能学习稀疏图中的表示实体向量模型,能有效将这些实体关系模型进行预测。(3)知识库中的实体属性特征和关系路径特征能相互影响,对于部分关系路径表达力差的特征,好的实体关系路径特征能进行模型惩罚,减少这样关系特征的权重,而对于关系路径表达力强的特征,在加入实体属性特征后能增加这些特征的权重。(4)除了传统的基于逻辑回归、支持向量机算法,通过优化hinge损失函数,log损失函数学习二分类模型,本研究展示了其他学习模型的优势。通过基于pairwise的模型优化方法,本研究能将一组正负实体对作为一个整体序列进行模型优化,学习这些正负实体对的秩序,从而将知识库补全作为一个排序结果进行训练。这种算法的另外一个优势就是能避免模型的正负例不平等问题,传统的算法随机选择若干比例的负例进行模型优化,而基于排序的算法可以将三元组中多种正例三元组排在负例三元组之前,提高模型的可用性。


\section{未来研究展望}
尽管本研究基于符号逻辑算法开始,拓展了关系路径和实体属性等不同类型的特征,增加了模型预测的优化方法,优化了目标函数,但是未来的研究还是有很多重要的工作,如何结合符号逻辑算法和表示学习算法,如何将表示学习中的优化目标函数和符号逻辑中的优化目标函数相互统一。

从知识库补全的关系路径开始拓展,本研究在路径排序算法中加入子图特征路径,拓展知识库的关系路径特征。同时也考虑了实体属性特征在知识库补全中的重要应用,以及如何结合关系路径和实体属性特征,从而增加模型的可预测性。另外,本研究也拓展了不同的优化目标函数,从实验中发现结合pairwise的知识库补全算法可以有效的优化一组正负实体对集合,很好的解决知识库中正负实例对不平衡的问题。

本研究未来希望能关注如何有效结合关系路径特征、实体属性特征、实体向量特征,不仅仅通过不同的算法抽取关系路径特征,转换不同类型实体属性,而且构建了深度学习模型,将知识库补全算法的不同特征进行融合,更好提高了机器学习模型的效果。通过将这些不同的实体、关系预测统一构建一个合理有效的目标函数,优化目标函数获得更好的模型系统,从而将知识库补全算法进行统一优化、改进,从而模型预测变得更加智能。

本研究未来关注的另外一个重要如何结合关系路径特征、实体属性特征、实体向量特征预测实体的一些属性值。尽管很多知识库中,如何预测实体和实体之间的关系是一个重要的问题,但是我们在实验中也非常明显的发现,很多实体的属性值都是缺失的,如很多重要人物的出生日期、很多地点的地理位置等信息存在明显的缺失或者错误,如何从实体、实体属性类型预测实体属性值,对于解决实体属性值的缺失或错误问题,有着重要的影响。这个问题对于未来的知识库补全、知识库拓展和应用也都是十分重要的问题,对于本人的未来研究之路,也是我需要进一步思考和探索的。

除了在理论和算法方面的研究工作之外,进行关系路径的知识库补全算法的一个不足时对于实验机器要求较高。通常需要耗费大量的内存来构建图,查找图中的关系路径,如何能降低关系路径搜索中对于机器硬件的要求也是一个可以优化的方向,如果能有更高效,内存使用更合理的随机游走路径搜索算法,也是很多领域关注的热点问题。