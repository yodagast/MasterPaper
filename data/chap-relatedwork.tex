% !Mode:: "TeX:UTF-8"
%%% Local Variables:
%%% mode: latex
%%% TeX-master: t
%%% End:

% 绪论和个人工作

\chapter{知识库补全研究现状}
\label{cha:related-work}

当前知识库补全相关的研究主要分为三个方面:(1)知识库的构建,各种基于互联网服务知识库构建、各个领域内的知识库构建、通用的知识库构建。(2)知识库中关系机器学习,基于已经构建好的知识库预测实体和实体之间的关系,进行知识库补全推理等。(3)知识库系统应用。如基于知识库的问答系统,
基于知识库的信息检索系统等,以及在各个领域内如生物、金融、教育等领域的本体知识库应用等。
其中研究的热点之一就是知识库补全,即如何发掘知识库中实体和实体之间隐藏的关系。

现存的知识库系统如NELL、Knowledge Vault等,
尽管已经包含了大量的实体、关系和属性,但这些知识库中很多实体和实体之间的关系任然是缺失的。
例如一个简单的知识库中存在一些三元组如:(北京师范大学,位于,北京)、(北京,位于,中国),
但是当预测北师大是否位于中国时,由于系统中并未存在(北京师范大学,位于,中国)这样直接的三元组,通常计算机很难直接进行这类常识性的推理。
如何构建一个辅助计算机进行常识推理的系统,发现知识库中隐藏的实体和实体之间的关系,是知识库补全系统的重要任务之一。
当前有很多基于表示学习和基于符号逻辑的知识库补全算法被提出,
这类算法都是基于知识库中现有的关系类型去预测知识库中未知的关系。
但是在知识库中存在大量的属性事实型三元组未被使用,而这些属性事实三元组在进行关系预测时有很重要的作用,
如对于已知的关系型三元组(A,isMarriedTo,B)、(A,HasChild,C),当预测B是否是C的父母时,这些相关实体的年龄、性别等实体属性也很重要,
如何能处理不同类型的实体属性特征,将其用于知识库补全系统中,是知识库补全工作的一个重要探索点。

在构建知识库补全系统的过程中,基于符号逻辑的算法通常通过训练一个二分类的分类器来判断实体和实体之间是否具有某种关系。通常这个分类器的正例是通过抽取现有知识库中存在的三元组作为正例,或称之为正例三元组,而负例则是采用随机采样等方法生成负例三元组。通常在知识库补全系统中,每个正例三元组可以构建上百组不同种类的负例三元组,如何解决正负例三元组不均衡的问题,正确预测实体和实体之间的关系,也是知识库补全算法的一个重要优化方向。

\section{知识库构及其应用}

\textbf{知识库构建} 随着信息技术的发展和应用,过去几十年中,人们采用不同的方法构建知识库,将人们的经验、
常识和现实世界中的事实知识用知识库的方法进行表示存储。
通常知识库可以由三元组组成。知识库的构建可以有多种方法:
(1)通过机器学习和自然语言处理技术自动抽取三元组\cite{Weikum2010FromIT}进行构建。
(2)半自动抽取的方法,通过从维基百科等网站的infobox基于规则模式抽取三元组。
(3)基于协同创作\cite{Vrandecic2014WikidataAF}的方法,通过众包的形式系统创建知识库。
(4)基于专家知识构建的领域内知识库,这些领域知识库有较高的专业性。
数据完整性、数据精确性和数据质量是考量这些构建知识库质量的重要指标,也是衡量知识库好坏的重要标准。


\textbf{YAGO知识库} YAGO\cite{Suchanek2007YagoAC}是一个从维基百科上抽取的、包含地理名词、WordNet\cite{Kasneci2008TheYA}等数据的知识库。
YAGO将WordNet的词汇定义与维基百科的分类体系进行了融合集成\cite{Suchanek2008YAGOAL},并集成了多种地理类型的数据,使得YAGO具有更加丰富的实体分类体系。YAGO还考虑了时间和空间知识\cite{Hoffart2011YAGO2EA},为很多知识条目增加了时间和空间维度的属性描述。目前,YAGO包含超过1.2亿条结合关系三元组和实体属性三元组的知识。YAGO也是IBM Watson\cite{Ferrucci2010BuildingWA}的后端知识库之一。
YAGO2\cite{Hoffart2013YAGO2AS}是YAGO的一个实例,当前YAGO2包括超过260万的实体和超过1.2亿的关系三元组知识,我们使用了其中实体的关系型三元组和属性型三元组共有4,484,914条、
37种关系型三元组的事实描述,同时有3,353,659条、35种属性性三元组的事实描述。
最新的YAGO3知识库\cite{Mahdisoltani2015YAGO3AK}进一步拓展了YAGO知识库,
集成了超过10种语言中的维基百科知识,属性值的准确率超过95\%。

\textbf{Freebase知识库} Freebase\cite{Bollacker2008FreebaseAC}是一个协同创作的知识库系统,内容通过用户添加,所有的条目采用结构化的形式,将结构分为三次:领域-类型-主题,其中,每一个条目叫做一个主题,每个主题包含不同的属性类型,一些同类型的主题组成一个类型,所有相关的类型构成一个领域。
这种通过协同创作方式创建了结构化的人类知识,截止2007年Freebase包含1.25亿条三元组,超过4000种类型和超过7000种属性知识。一些基于Freebase研究结构化知识的问答系统\cite{Yao2014InformationEO}\cite{Yao2015LeanQA},关注Freebase中的信息抽取和语义解析\cite{Yao2014FreebaseQI},也有一些研究关注Freebase中的实体消除歧义问题\cite{Zheng2012EntityDW}。

表\ref{tab:addlabel-kb-list}展示了一些不同类型的知识库的实体、关系类型和三元组等数量,其中的M表示百万数量。
除了YAGO和Freebase知识库,也有很多重要的知识库如:Google Knowledge Graph,Wikidata,DBpedia等,这些不同类型的知识库都有着各自的特色和优势。
通过分析表中的不同实体关系三元组、关系属性三元组分析可以发现,知识库中的实体数量、关系类型数量、以及三元组数量都很多。
本研究中的实验考虑到同时使用属性三元组和实体关系三元组,从而选择在YAGO和Freebase数据集中筛选属性特征较多的实体,
构建两个子数据集进行模型训练。

% Table generated by Excel2LaTeX from sheet '知识库'
\begin{table}[htbp]
  \centering
  \caption{不同知识库数据统计}
    \begin{tabular}{|l|l|r|l|}
    \hline
    知识库   & 实体数量  & \multicolumn{1}{l|}{关系类型数量} & 三元组数量 \\
    \hline
    Knowledge Graph & 570M  & 35000 & 18000M \\
    \hline
    YAGO2 & 9.8M  & 114   & 447M \\
    \hline
    DBpedia & 4.6M  & 1367  & 538M \\
    \hline
    Wikidata & 18M   & 1632  & 66M \\
    \hline
    Freebase & 40M   & 35000 & 637M \\
    \hline
    \end{tabular}%
  \label{tab:addlabel-kb-list}%
\end{table}%


\textbf{知识库应用}
在通过自动、半自动的信息抽取、专家知识构建知识库后,知识库库系统在信息检索、广告、推荐和金融领域等
有着各种广泛的应用,知识库不仅能辅助人们检索、学习各种信息和知识,而且一些专家的领域内的知识有序化,规范化,
能更好帮助人们去了解学习一个领域的专业知识。

知识库最广泛的应用是谷歌和微软等商业公司的信息检索系统,2014年谷歌提出的知识图谱\cite{Dong2014KnowledgeVA}就是结合FreeBase、Wikidata以及各种互联网数据进行信息抽取、
知识整合后的著名系统,谷歌也在其博客上讲述了如何将知识库应用在实际的检索中,
微软的Satori也是一个类似的知识库系统,被必应搜索广泛应用。其他搜索公司如百度\footnote{https://kgopen.baidu.com/index}、搜狗\footnote{https://www.sogou.com/tupu/person.html}等公司都在知识库构建和信息检索领域有各自的应用。
对于每一个用户搜索的关键词,我们可以通过知识库系统来返回更丰富,更全面的信息。比如搜索一个企业法人的姓名,我们的智能搜索引擎可以返回与这个人相关的所有历史借款记录、联系人信息、行为特征等。另外我们通过可视化把复杂的信息以非常直观的方式呈现出来,使得用户对于对隐藏信息的来龙去脉一目了然。

基于知识库的问答系统也有广泛的发展,2000年就有人提出基于知识的问答系统\cite{HERMJAKOB2000KnowledgeBasedQA},通过构架知识库完成问答。
也有很多方法基于知识库进行表示学习\cite{Yang2014JointRE},通过学习实体的低维度表示,学习知识库的实体和实体之间的关系,然后构建相关的知识库问答系统。
还有一些基于标签语义解析的方法,构建知识库问答系统\cite{Yih2016TheVO}。
一些论文\cite{Yu2017ImprovedNR}研究结合深度学习、关系抽取和问答系统,从而获得更好的知识库问答效果。
一些论文\cite{KeySun2016OpenKB}研究基于开放的知识库构建开放领域的问答系统,期望能基于百科知识进行问答系统构建。
也有基于知识库的系统通过结合智能计算\cite{Pandey2009KnowledgeAI},在医疗诊断领域获得一些突破。
也有一些研究结合了知识库和医疗图像信息\cite{Halpern2014EvaluationOC},对一些医疗疾病进行诊断治疗,期望能获得更好的治疗效果。

知识库在金融、教育等领域也可以有各种有效的应用。在进行金融风险检测时,
我们可以使用不一致性验证可以用来判断一个借款人的欺诈风险,发现交易欺诈。比如借款人A和借款人B在同一地点、同一公司工作,但A和B的公司联系方式完全不一样,出现逻辑不通之处,则我们可以通过知识库补全推理发现这种交易欺诈。在金融领域除了贷前的风险控制,知识库也可以帮助我们在贷后进行催收。比如在贷后失联客户管理的问题上,知识库可以帮助我们挖掘出更多潜在的新的联系人,找到很多潜藏的知识库信息,发掘潜藏的关系等。

除了金融领域的知识库应用,在互联网和广告系统中,知识库也可以帮助我们分析用户和理解用户。通过结合多种数据源去分析实体之间的关系,发现用户的行为特征,更好的理解用户意图。比如一利用知识库来分析用户之间的关系,去发现一个组织的共同喜好,对于有相同特征的用户群体,制定合理有效的营销策略。通过对用户关系的分析,知识库帮助我们更好的、更深入的理解用户的需求,提升广告、搜索\cite{Liu2005KnowledgebasedQE}、推荐\cite{Burke2000KnowledgebasedRS}等相关领域的效果。

\section{基于表示学习的知识库补全}
\label{cha:presentation}

当前知识库补全方法主要有两种方法:基于表示学习的知识库补全和基于符号逻辑的知识库补全。
基于表示学习的方法是通过学习实体和关系的低维度向量表示,
用向量的相似度计算预测实体之间的关系。常见的表示学习方法包括基于神经网络翻译的预测算法如TransE\cite{NIPS2013_5071}、TransH\cite{Wang2014KnowledgeGE}、TransR\cite{Huang2017ImprovedKB}等,也有基于矩阵张量分解、协同过滤的表示学习方法如RESCAL\cite{Nickel2011}等。

基于学习表示的知识库补全技术是近年来的研究热点之一,近年来获得了极大的关注热度,
这类知识库补全技术通过不同的目标函数,希望能学习到知识库中实体和关系的低维度向量表示。
这些实体的向量维度通常是50-300维度之间,将高维度稀疏的图数据张量,学习获得低维度的连续向量表示。
获得实体和关系的向量表示后,可以通过向量之间余弦距离计算实体-关系-实体相似度。目前这类表示学习算法包括两类,基于神经网络翻译模型的表示学习和基于协同过滤的表示学习算法。

\textbf{协同过滤学习模型} 基于协同学习的知识库补全算法借鉴了推荐系统中的协同过滤模型,将基于用户-商品的协同矩阵分解技术进行拓展,
构建了基于实体-关系-实体的三维张量模型,进行矩阵分解和实体关系学习预测。
\textbf{RESCAL}是一种基于表示学习的潜藏特征模型,基于矩阵分解、协同过滤进行知识库补全,
RESCAL通过构建三维张量分解技术进行知识库补全,将实体-关系-实体构建三维的张量矩阵,提供了一种有效的表示学习方法。
和其他表示学习模型不同,RESCAL这类模型更多借鉴推荐系统中张量分解算法,通过偏最小二乘算法,随机梯度下降优化算法,
训练学习实体和关系的向量表示,学习实体和实体之间的相似度,进行对实体和实体之间未知的关系进行预测。
详细来说,RESCAL基于R阶的分解模型,$\chi_k$~(\ref{equ:chap:feature-chi})被分解。
\begin{equation}
\label{equ:chap:feature-chi}
\chi_k \approx AR_{k}A^T, k =1...m
\end{equation}
其中A表示一个$n\times r$的矩阵,表示n个实体和r种关系组成矩阵,$R_{k}$是一个$r\times r$的矩阵。
则只基于随机梯度下降模型,最小化函数:
$\min \limits_{A,R_k} f(A,R_k)+g(A,R_k)$
其中$f(A,R_k)$是度量$\chi_k$和$AR_{k}A^T$距离的函数。
$g(A,R_k)$是$A,R_k$的复杂度惩罚项。通过梯度下降等学习算法可以计算获得模型的参数。

其他类似基于协同过滤模型表示学习进行知识库补全的方法还有张量分解机算法\cite{Rendle2010FactorizationM},
基于隐藏变量的张量分解机算法\cite{Rendle2012FactorizationMW},结构化的低维表示\cite{2009EmbeddingLS}、无结构化的低维表示等。
基于张量分解模型和推荐系统等领域的协同过滤算法相似,都是从矩阵补全角度协同过滤学习向量模型。
不同的是在知识库中,矩阵其实是由实体-关系-实体组成的三维张量,而推荐系统中的用户-商品矩阵是二维向量。
一些实验\cite{Dong2014}表明在稀疏的知识库图中,使用张量分解模型有较好的预测效果。

\textbf{神经网络翻译模型} 基于神经网络翻译模型的知识库补全算法近年来获得了极大的关注,
很多基于神经网络翻译模型的知识库补全算法被提出如TransH、TransR等。
\textbf{TransE}是一种典型的基于神经网络翻译模型的机器学习模型,
研究考虑如何将知识库中的多种不同的关系、实体,
学习获得它们的低维向量表示,期望能获得$h+r\approx t$效果,其中h和t分别是头实体和尾实体
学习到的低维向量,r是头实体和尾实体之间的关系向量表示。TransE模型定义了间隔损失函数L~(\ref{equ:chap:feature-transe}):
\begin{equation}
\label{equ:chap:feature-transe}
L=\sum_{(h,r,t)\in S} \sum_{(h',r,t')\in S'}[\gamma+d(h+r,t)-d(h'+r,t')]_{+}
\end{equation}
其中$\gamma$是间隔超参数,s是真实存在的知识库,$(h,r,t)$是这个知识库中的三元组
,$s'$是负例三元组组成的知识库实体,$(h',r,t')$表示负例三元组。$d(h+r,t)$表示头实体加上关系向量
和尾实体之间的欧氏距离。通过定义间隔损失函数,TransE期望学习的模型能保证正实例对比负实例对距离小,
这样就能保证正实例对比负实例对余弦相似度更高。

其他的算法如TransH和TransR都是基于TransE模型的基础上,通过学习不同类型、不同目标的损失函数来获得低维度向量表示。
TransH考虑一些关系中的映射属性知识,如一对一、一对多、多对多关系等,训练学习这些关系的低维度向量表示。
TransR\cite{Lin2015LearningEA}考虑实体和关系的多方面属性,在分割独立的向量空间中学习实体、关系的低维度向量表示。
ProjE\cite{Shi2017ProjEEP}基于神经网络的实体关系投影表示学习。
这些不同类型的基于翻译模型的神经网络表示学习,能借助神经网络的优势,在CPU、GPU、TPU等不同的底层硬件中进行模型训练,从而能进行大规模的知识库表示学习和预测补全。

\section{基于关系路径的知识库补全}
\label{cha:symbolic}

基于关系路径的知识库补全算法也是多年来知识库补全、知识库推理中的关注热点之一。从90年代昆兰等人提出的规则归纳,
将观察集数据中的知识以规则的形式提炼出来,到近年来热门的基于\textbf{路径排序算法(PRA)}和\textbf{子图特征抽取(SFE)}的知识库推理补全。这类进行知识库补全的算法
多是利用符号逻辑,精确使用知识库中的关系路径特征,统计发现知识库图数据结构中存在的规则或规律,从而进行知识库补全预测。
基于符号逻辑的方法主要包括AMIE\cite{Galrraga2013AMIEAR}、PRA\cite{Lao2010}和SFE\cite{Gardner2015}等;
其中,AMIE方法通过从知识库中挖掘关联规则进行知识库补全,通过发掘知识库中频繁出现的关系路径,
从而发现知识库中实体和实体之间隐藏的关系。
路径排序算法PRA抽取连接实体和实体之间的关系路径,通过关系路径作为特征向量来预测实体之间的关系,
子图特征抽取SFE在PRA框架的基础上,抽取更多隐藏的关系路径特征进行模型预测。

\textbf{规则挖掘算法} 规则挖掘是基于训练集知识库中的数据,
通过挖掘知识库中隐藏的关联规则,进行知识发现和知识库补全。早期昆兰等人
提出了一阶逻辑推理的算法FOIL\cite{Quinlan1993FOILAM},通过学习知识库中的例子来构建霍顿子句。如公式~(\ref{equ:chap:rw-foil})就是一个典型的一阶逻辑推理。
实体A是实体C的母亲,实体A和实体B结婚,则可以推断实体B是实体C的父亲,尽管这种推理并不是百分之百准确。
\begin{equation}
\label{equ:chap:rw-foil}
MotherOf(A,C)\land MarriedTo(A,B)\Rightarrow FatherOf(B,C)
\end{equation}
FOCL\cite{PAZZANI1991DetectingAC}等算法也是基于一阶逻辑推理构建的知识库搜索算法。
近年来随着互联网的快速发展,Schoenmackers\cite{Schoenmackers:2010}等人也将一阶逻辑逻辑推理用于互联网文本中。
其他规则挖掘的典型算法包括AMIE\cite{Galarraga2013},AMIE受关联规则启发,基于开放世界假设,在一阶逻辑推理的基础上,
能更快更高效的处理大规模的知识库数据,作者此后对AMIE算法进行改进获得AMIE+\cite{Galrraga2015FastRM},这些算法在
YAGO知识库中获得了很好的效果。RDF2Rules\cite{Wang2015RDF2RulesLR}等算法也是基于逻辑推理,构建频繁的谓词圈,
获得了更加高效和准确的逻辑推理算法。

\textbf{路径排序算法} 路径排序算法\cite{Lao2010}是2010年,
劳逆等人提出的知识库补全算法。在传统的一阶逻辑推理的基础上,
路径排序算法基于随机游走,在由知识库构成的图数据中查找更多、更长的有效的路径进行知识库推理,
和规则学习方法不同,路径排序算法在抽取图中的关系路径后,构建了分类学习器,通过分类器学习不同关系路径的权重。
利用每种关系下三元组的不同关系路径权重,来进行新关系事实的预测,从而构建了新的知识库补全算法。此外,作者还构建了基于反向
的随机游走图搜索算法\cite{Lao2015LearningRF},进一步提升路径排序算法在知识库补全中的效果。考虑给定一个知识库关系$r_i$,
我们抽取所有具有这种关系的实体对构成集合$R_i$~(\ref{equ:chap:rw-ri})
\begin{equation}
\label{equ:chap:rw-ri}
R_i=\{{(h_{ij},t_{ij}),(h_{ij},r_i,t_{ij})\in KB}\}
\end{equation}
随机游走查找路径后,我们抽取所有连接实体对$(h_{ij},t_{ij})$的路径类型$p_i$~(\ref{equ:chap:rw-pi})构成分类特征:
\begin{equation}
\label{equ:chap:rw-pi}
P_i=\{{p_i|(h_{ij},p_i,t_{ij})\in KB}\}
\end{equation}
从而我们可以选择合适的分类器模型进行关系路径的训练学习,利用合适的路径类型来进行关系预测,知识库补全。

子图特征抽取\cite{Gardner2015}提供了一种更加简便有效的关系路径计算方法。相比于路径排序算法,子图特征抽取能获得更多的潜藏路径,
在进行关系预测的时候能获得更加显著有效的提升。此外,也有很多算法结合了子图特征抽取和表示学习算法\cite{Gardner2014}进行知识库补全。
Coupled PRA\cite{Wang2016}算法对相似的关系进行聚类,
对于处理包括成百上千条关系的知识库,需要将相互影响的关系进行聚合,
构建多任务的路径排序模型。


\section{知识库补全模型评价}
知识库补全的正例实体对可以选择知识库中已经存在的实体对,而构建负例实体对需要一些规则来生成负例三元组。
此外,知识库补全模型一方面可以看做是正负例实体对分类的过程,从而可以选择分类器的模型评价函数,另一方面,知识库补全也可以看做对正负例实体对进行排序,从而可以选择排序模型评价函数。以往的知识库补全算法通常只选择分类评估指标和排序评估指标中的一类,本研究选择两种类型的评估指标进行模型评估。

\textbf{知识库补全假设} 知识库补全系统中,通常需要假定知识库中三元组正负实例对的正确性,或者在何种限制模式生成三元组负例。常用的有三种假设:(1)\textbf{开放世界假说},(2)
\textbf{封闭世界假说},(3)\textbf{局部封闭世界假说}。我们给定一个三元组集合$D^+$~(\ref{equ:chap:rw-d})表示正例:
\begin{equation}
\label{equ:chap:rw-d}
D^+=\{(h,r,t)|(h,t) \in KB\}
\end{equation}
对于开放世界假说,给定一个知识库中的三元组集合,我们认为这些实际存在三元组是正例三元组,对于在知识库中不存在的三元组,
开放世界假说认为这个三元组是不确定的,可以通过概率大小预测这个三元组的正确性。

对于封闭世界假说,给定一个知识库三元组集合,我们认为只有知识库存在的三元组才是正例三元组。对于知识库中不存在的三元组,
封闭世界假说认为这些三元组是负例三元组。但通常这样会产生正负实例极其不平衡情况,每个知识库中存在三元组都能生成数百例
负例三元组。这些负例三元组组成负例集合$D^-$~(\ref{equ:chap:rw-dminus}):
\begin{equation}
\label{equ:chap:rw-dminus}
\begin{split}
D^-=\{(h_j,r,t_i)|(h_i \ne h_j \wedge (h_i,t_i) \in KB)\} \\
 \cup \{(h_i,r,t_j)|(t_j \ne t_i \wedge(h_i,t_i) \in KB)\}
 \end{split}
 \end{equation}

局部封闭世界假说对封闭世界假说进行了改进。给定一个关系,知识库中这个关系的三元组记为正例三元组,随机在这个关系中
替换头尾实体,生成这个关系下实体集合中错误的实体对,这样就可以大大减少所有负例三元组个数。基于局部封闭世界假设
构建的负例三元组集合$D^-$~(\ref{equ:chap:rw-dmin})可以用如下公式表示:
\begin{equation}
\label{equ:chap:rw-dmin}
\begin{split}
D^-=\{(h_{ik},r_i,t_{ij})|(h_{ik} \ne h_{ij} \wedge (h_{ik},t_{ik}) \in KB  \wedge (h_{ij},t_{ij}) \in KB)\} \\
 \cup \{(h_{ij},r_i,t_{ik})|(t_{ij} \ne t_{ik} \wedge(h_{ik},t_{ik}) \in KB \wedge(h_{ij},t_{ij}) \in KB)\}
 \end{split}
\end{equation}

\textbf{模型评价方式} 知识库补全效果的评价方法借鉴了信息检索和分类器模型等研究中的模型评价方式,主要采用MAP和MRR\cite{Gardner2014}指标进行模型评价。考虑到在一个给定关系下,给定一个头实体和一系列的候选尾实体,我们希望排在前面的实体对是正确的实体对,
如果正例实体对排在负例实体对后面,则这个模型的预测效果有待改进。
所以我们借鉴了信息检索领域相关的评价方法,除了采用关键的MAP评价指标,
我们也采用了一些常见的AUC和Hit@1评价指标,希望能详尽的分析模型的有效性和可预测性。
本部分介绍了四类评价知识库补全的模型评估指标,包括:(1)hit@1,描述排序在第一位的正例三元组比例,(2)平均准确率MAP,衡量所有正例三元组在整个排序列表中的排序情况,值越大说明整体排序情况越好,(3)平均秩倒排序MRR,每组三元组中,正例三元组秩序排名,值越大越好;(4)AUC,使用二分类模型评价函数来预测知识库补全效果。

在信息检索中,hit@1表示每组三元组对应的正负实例中,正实例在该组正负例三元组中排名第一的三元组比例,
尽管这种评价指标较为简单,但可以有效衡量模型中正例三元组排在第一位比例。对于每个关系,
我们定义的的hit@1~(\ref{equ:chap:rw-hit1})计算公式如下:
\begin{equation}
\label{equ:chap:rw-hit1}
hit@1=\frac{\sum_{q=1}^n{hit_q}}{n}
\end{equation}
其中$hit_q$表示该关系下第$q$组实体对预测结果中,正例是否排名第一,
如果正例排名第一则$hit_q$为1,否则为0。

\textbf{平均准确率(mean average precision)}给定一个头实体和一系列候选尾实体,头尾实体对之间的关系是否存在作为正负例,
我们采用局部封闭世界假设,在知识库中存在的实体对标记为1,即为正例,否则为负例。
MAP核心是利用头实体对应的相关的尾实体出现的位置来进行排序算法准确性的评估。
它反映系统在全部相关文档上性能的单值指标。系统模型计算出来的实体对越靠前(rank 越高),MAP就应该越高,否则准确率默认为0。MAP~(\ref{equ:chap:rw-map})的计算公式如下,对于每一组正负例实体对,
AP~(\ref{equ:chap:rw-ap})表示一个头实体和对应的候选尾实体(或者一个尾实体及其对于候选头实体)的评价得分。
\begin{equation}
\label{equ:chap:rw-map}
MAP=\frac{\sum_{q=1}^nAP(q)}{n} \\
\end{equation}
\begin{equation}
\label{equ:chap:rw-ap}
AP(q)=\frac{\sum_{i=1}^kP(i)\times rel(i)}{k}
\end{equation}

其中MAP表示n个头实体和其候选尾实体的平均准确度的均值。AP(q)是一个头实体和其K个候选尾实体排序结果的评价,
如果正确的候选尾实体排在错误的候选尾实体之前,则AP值越高;
如果所有的头实体和其候选尾实体AP值越高,则模型的MAP值也就越高。
$rel(i)$在第i个实体对是预测结果时为1,否则为0。


\textbf{平均倒排秩序(mean reciprocal rank)}通过定义第一个正确的候选尾实体来判断模型好坏,一个模型的MRR值越高,说明正确的候选尾实体在模型中排序越靠前。
MRR通常和MAP一起综合评价一个模型的好坏。
MRR~(\ref{equ:chap:rw-mrr})的定义公式如下:
\begin{equation}
\label{equ:chap:rw-mrr}
MRR=\frac{1}{n} \sum_{i=1}^n \frac{1}{r_i}
\end{equation}
其中n表示一个关系下所有不同头实体的数目,$r_i$是每个头实体对应的正确尾实体的秩序。
MRR在很多基于问答系统的知识库补全系统\cite{West2014}中也是常见的评价方法之一。
通常情况下,基于符号逻辑的知识库补全系统中MRR值较高,多数实验结果都在0.9左右,大部分关系中的MRR实验结果差别不显著,
所以本研究的重点是关注如何显著提高知识库补全系统中关系补全的MAP指标。


除了常见的排序模型评价方式,我们在实验过程中也引入二分类分类器模型进行模型评价,
同时也考虑模型对正负例排序能力。AUC(Area Under Curve)被定义为操作者曲线,或者称为ROC曲线(tpr和fpr曲线函数)下的面积,显然这个面积的数值不会大于1,同时这个值一般大于0.5。
其中真阳性(tpr)~(\ref{equ:chap:rw-tpr})和假阳性(fpr)~(\ref{equ:chap:rw-fpr})的计算公式如下所示:
\begin{equation}
\label{equ:chap:rw-tpr}
tpr=\frac{tp}{tp+fn}
\end{equation}
\begin{equation}
\label{equ:chap:rw-fpr}
fpr=\frac{fp}{fp+tn}
\end{equation}
随机挑选一个正样本以及一个负样本,当前的分类算法根据计算得到的打分值将这个正样本排在负样本前面的概率就是AUC值。对所有实体对的预测得分进行排序,选择不同的阈值对正负例进行切分计算真阳性tpr和假阳性fpr的得分,从而绘制出完整的ROC曲线,并计算得到AUC的面积大小。AUC取值越大,当前的分类算法越有可能将正样本排在负样本前面,即能够更好的分类,整个模型的排序效果也更好。操作者曲线面积AUC~(\ref{equ:chap:rw-auc})计算公式如下所示:
\begin{equation}
\label{equ:chap:rw-auc}
auc=\int_{0}^{1}{tpr}\times d{(fpr)}
\end{equation}

在知识库补全预测中,无论是逻辑回归模型、支持向量机模型还是学习排序模型,
都需要关注模型的二分类的分类器性能问题。对于知识库中的实体对排序、实体对打分预测,
在评估一组实体对集合的排序秩序性能之外,也关注模型的分类性能好坏,
这些对于知识库补全算法是十分必要的。

\section{本章工作总结}

本部分介绍了当前知识库技术相关领域的一些重要研究内容,从知识库构建技术到知识库应用和发展,
我们学习研究了各种重要的知识库构建、推理、应用相关内容,介绍了如何评价知识库补全算法的相关指标。本部分重点关注了知识库补全算法相关的研究现状,基于符号逻辑和基于表示学习的知识库补全算法一直是相关领域的研究热点,本部分抓住这些研究中的重要公式、推理和相关假设,为构建我的知识库补全算法框架提供了坚实的基础。







