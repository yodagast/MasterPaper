% !Mode:: "TeX:UTF-8"
%%% Local Variables:
%%% mode: latex
%%% TeX-master: t
%%% End:
%\secretlevel{绝密} \secretyear{10}


\ctitle{基于关系路径的知识库补全算法研究}
%\makeatletter
%\cdegree{博士}

\makeatother

%\makeatletter
%\cdegree{博士}

\makeatother

\cauthor{黄勇}
\csupervisor{王志春\;副教授 \\ \hspace*{3em}}
\cdepartment[]{计算机软件与理论} 
\cmajor{知识工程}
\cnum{201521210022}
\cdate{\the\year  年 \the\month 月}

\etitle{Knowledge Base Completion by relational path methods}

% 定义中英文摘要和关键字
\begin{cabstract}
  知识库通常从非结构化、半结构化的数据中抽取事实,构建结构化的知识,常用三元组进行存储表示。知识库在信息检索、问答系统和人工智能等不同领域具有重要作用,尽管当前有很多大规模知识库如Freebase、YAGO包含大量三元组,但这些知识库中的三元组是不完备的,很多实体对之间的关系存在缺失。 
  知识库补全基于现有知识库的三元组,预测知识库中实体和实体之间的关系。
  当前的知识库补全算法主要包括基于符号逻辑的知识库补全算法和基于表示学习的知识库补全算法,
  其中一种典型的基于符号逻辑的知识库补全算法是\textbf{路径排序算法}。路径排序算法抽取
  实体之间的\textbf{关系路径特征},构建逻辑回归分类模型,对知识库中实体和实体之间关系进行预测。

  当前路径排序算法基于关系路径进行知识库补全,并没有考虑实体和实体之间的属性特征关系,一些知识库补全系统构建的正负实例对比例悬殊,进行模型训练很难抽取有效的路径特征,也难以进行模型训练,如何构建更丰富有效的特征类型,学习更有效的模型,任然是知识库补全的重要任务。
  本论文在传统的关系路径特征基础上,提出了一种新的结合关系路径特征和\textbf{实体属性特征}知识库补全模型,并将不同类型的关系路径特征、实体属性特征进行组合,显著提升了知识库补全效果。
  考虑到知识库补全中正负例不平衡问题,为了获得更好的模型预测结果,在传统逻辑回归模型的基础上,本研究基于\textbf{学习排序模型}的知识库补全,通过学习排序损失函数,优化正负实体对的排序,从而提高知识库补全模型效果。本论文的研究创新点主要有:
  \begin{itemize}[$\bullet$]
    \item 抽取关系路径和实体属性特征作为知识库补全的特征,将多种不同类型特征进行组合,极大拓展了知识库补全系统的维度,增强了知识库补全效果。
    \item 基于学习排序算法预测知识库补全中实体和实体的关系,通过直接学习知识库补全中的实体对排序顺序,改进损失函数进行模型预测。
    \item 构建基于逻辑回归、学习排序树模型方法的知识库补全模型,通过候选实体对生成,模型排序,利用不同知识库数据特征选择合适的排序模型。
  \end{itemize}
\end{cabstract}

\ckeywords{知识库补全,路径排序算法, 学习排序模型,关系路径特征,实体属性特征}

\begin{eabstract}
Knowledge base (KB) completion aims to predict new facts
from the existing ones in KBs. There are many KB completion approaches,
one of the state-of-art approaches is Path Ranking Algorithm
(PRA), which predicts new facts based on relational path types connecting entities.
PRA takes the relation prediction as a classification problem, and
logistic regression or SVM is used as the classification model. In this paper, our model
use both literal facts and relational path types as feature matrix, which is much more comprehensive and complicated. We
consider the relation prediction as a ranking problem, learning to rank
model is trained on relations to predict new facts. Besides. We propose to extract literal features from literal facts, and incorporate them with path-based
features extracted from relational facts; predictive
model is then trained to infer new
facts with assembly features and bring higher precision scores in classification metrics and ranking metrics. Experiments on YAGO show that our proposed approach outperforms approaches using relational path features and classification models.
This paper has three main contribution, including:
\begin{itemize}[$\bullet$]
    \item KB completion feature type is comprehensive including both relational path and literal features to generate feature matrix, and in different KBs our features help to generate higher scores both in ranking tasks and classification tasks.
    \item KB completion was considered as a ranking method, we use learning to ranking model to rank entity pairs rather than simple taking it as a classification model.
    \item KB completion methods are easily explainable, with combined literal and relational facts we can easily predict relation in KBs, tree models are used with assembly features, this helps us to rank entity pairs in different.
  \end{itemize}

\end{eabstract}

\ekeywords{Knowledge Base Completion, Learning to Rank, Path Ranking, Relational Path}
