% !Mode:: "TeX:UTF-8"
%%% Local Variables:
%%% mode: latex
%%% TeX-master: t
%%% End:
%\secretlevel{绝密} \secretyear{10}

\ctitle{基于关系路径的知识库补全算法技术研究}
%\makeatletter
%\cdegree{博士}

\makeatother

\cauthor{黄勇}
\csupervisor{王志春\;副教授 \\ \hspace*{6em}}
\cdepartment[]{计算机软件与理论}
\cmajor{知识工程}
\cnum{201521210022}
\cdate{\the\year  年 \the\month 月}

\etitle{Knowledge Base Completion by Symbolic methods}

% 定义中英文摘要和关键字
\begin{abstract}
  知识库补全基于现有知识库数据,预测知识库中实体对之间新的关系。当前有许多知识库补全算法,
  其中效果较好的一种基于逻辑符号的知识库补全算法是路径排序算法(PRA)。路径排序算法基于
  实体之间的关系路径特征,构建逻辑回归分类模型,进行知识库中实体对之间关系预测。本研究
  基于路径排序算法,在传统的关系路径特征基础上,抽取关系路径特征和实体属性特征,将这两种不同类型的
  路径特征进行组合,极大的丰富了关系预测的属性特征。
  其次考虑到知识库补全中正负例不平衡问题,为了获得更好的模型预测结果,在传统逻辑回归模型的基础上,  本论文研究了基于学习排序算法的路径补全模型,通过学习排序算法的损失函数,进行知识库关系补全预测。
  本毕业论文的研究创新点主要有:
  \begin{itemize}[$\bullet$]
    \item 抽取关系路径和实体属性作为知识库补全的特征,将两种不同类型特征进行组合,极大增强了知识库补全系统的特征维度。
    \item 基于学习排序算法预测知识库补全中实体对的关系,通过直接学习知识库补全模型中的排序损失函数进行模型预测。
    \item
        构建了基于逻辑回归排序、树方法排序,基于深度神经网络排序算法的知识库补全模型,可以利用不同知识库数据特征选择合适的排序算法模型。
  \end{itemize}
\end{cabstract}

\ckeywords{知识库补全,路径排序算法, 学习排序,符号逻辑}

\begin{eabstract}
Knowledge base (KB) completion aims to predict new facts
from the existing ones in KBs. There are many KB completion approaches,
one of the state-of-art approaches is Path Ranking Algorithm
(PRA), which predicts new facts based on path types connecting entities.
PRA takes the relation prediction as a classification problem, and
logistic regression or SVM is used as the classification model. In this paper, we
consider the relation prediction as a ranking problem, learning to rank
model is trained on relations to predict new facts. Besides, our learning to rank model
use both literal and relational facts as feature matrix.
We propose to extract literal features from
literal facts, and incorporate them with path-based
features extracted from relational facts; predictive
model is then trained to infer new
facts with much more comprehensive features and bring higher precision scores. Experiments on YAGO show that our proposed approach outperforms approaches using
relational features and classification models.
This paper has three main contribution:
\begin{itemize}[$\bullet$]
    \item KB completion feature type is much more comprehensive, we use both literal and relational facts to generate feature matrix.
    \item KB completion was considered as a ranking method, we use state-of-art method such as learning to ranking model to rank entity pairs
    \item KB completion methods are explainable, with combined literal and relational facts we can easily predict relation in KBs.
  \end{itemize}

\end{eabstract}

\ekeywords{Knowledge Base Completion, Learning to Rank, PRA, symbolic model}
